\documentclass[10pt]{article}
\usepackage[a4paper,margin=2cm]{geometry}
\usepackage[brazilian]{babel}
%\usepackage[utf8]{inputenc}
\usepackage[T1]{fontenc}
\linespread{1.3}
\parskip=12pt
\parindent=0pt
\usepackage{enumerate}
\usepackage{amsfonts}
\usepackage{amsmath}
\usepackage{amsfonts}
\usepackage{graphicx}

\begin{document}
	\begin{center}
		{\Large{\textbf{Lista 1 - Macroeconomia III 2017}}}\\
		\vspace{0.2cm}
		Professor: Ricardo Cavalcanti\\
		Monitora: K�tia Alves\\
		Alunos: Alexandre Machado e Raul Guarini
	\end{center}
	
\section*{Exerc�cio 1}

\section*{Exerc�cio 2}

\begin{enumerate}[i)]
\item Seja a fun��o $u(.)$ estritamente crescente. Isto implica que a restri��o de recursos do planejador vale com igualdade em todo instante do tempo. Dada a parametriza��o escolhida, isto � equivalente a $\gamma > 0$. Da�, � verdade que 
$$c_{t} = k_{t}^{\alpha} + (1-\delta)k_{t} - k_{t+1}$$ em todo per�odo, j� introduzindo o formato funcional da fun��o de produ��o. Deste modo, o problema pode ser visto como a escolha da sequ�ncia �tima de capital:
\begin{gather*}\max_{\{k_{t+1}\}_{t=0}^{\infty}} \left\{\sum\limits_{t=0}^{\infty} \beta^{t}u( k_{t}^{\alpha} + (1-\delta)k_{t} - k_{t+1})\right\} \\ 
\text{s.a } \begin{cases}
 	k_{t+1} \in [0, k_{t}^{\alpha}]\\
 	k_{0} \text{ dado}
 \end{cases}
\end{gather*}

\item A vari�vel de estado relevante � o capital atual, sendo o consumo atual e o capital do pr�ximo per�odo as vari�veis de controle. Entretanto, a restri��o de recursos, ao valer com igualdade, nos permite lidar apenas com uma vari�vel de controle, a saber, o capital do pr�ximo per�odo. Formula��o recursiva:
\begin{gather*}
	V(k) = \max_{k'} \{u(k^{\alpha} + (1-\delta)k - k') + \beta V(k')\} \\
	\text{s.a } k' \in [0, k^{\alpha}]
\end{gather*}

\item O operador de Bellman nesse caso � dado por 
$$ T[V](k) = \max_{k'} \{u(k^{\alpha} + (1-\delta)k - k') + \beta V(k')\}$$
A solu��o do problema do consumidor consiste num ponto fixo deste operador.

\item C�digo anexo.
\item C�digo anexo.

\end{enumerate}

\section*{Exerc�cio 3}

\section*{Exerc�cio 4}

\section*{Exerc�cio 5}

\section*{Exerc�cio 6}

\section*{Exerc�cio 7}
\end{document}